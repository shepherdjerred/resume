%-------------------------
% Resume in Latex
% Author: Sourabh Bajaj, modified by Jerred Shepherd
% License: MIT
%------------------------

\documentclass[letterpaper,11pt]{article}

\usepackage{latexsym}
\usepackage[empty]{fullpage}
\usepackage{titlesec}
\usepackage{marvosym}
\usepackage[usenames,dvipsnames]{color}
\usepackage{verbatim}
\usepackage{enumitem}
\usepackage{fancyhdr}
\usepackage[english]{babel}
\usepackage{tabularx}
\usepackage{xcolor}
\usepackage{hyperref}

\pagestyle{fancy}
\fancyhf{} % clear all header and footer fields
\fancyfoot{}
\renewcommand{\headrulewidth}{0pt}
\renewcommand{\footrulewidth}{0pt}

% Adjust margins
\addtolength{\oddsidemargin}{-0.5in}
\addtolength{\evensidemargin}{-0.5in}
\addtolength{\textwidth}{1in}
\addtolength{\topmargin}{-.5in}
\addtolength{\textheight}{1.0in}

\urlstyle{same}

\raggedbottom
\raggedright
\setlength{\tabcolsep}{0in}

% Sections formatting
\titleformat{\section}{
  \vspace{-4pt}\scshape\raggedright\large
}{}{0em}{}[\color{black}\titlerule \vspace{-5pt}]

%-------------------------
% Custom commands
\newcommand{\resumeItem}[2]{
  \item\small{
    \textbf{#1}{: #2 \vspace{-2pt}}
  }
}

\newcommand{\resumeSubheading}[4]{
  \vspace{-1pt}\item
    \begin{tabular*}{0.97\textwidth}[t]{l@{\extracolsep{\fill}}r}
      \textbf{#1} & #2 \\
      \textit{\small#3} & \textit{\small #4} \\
    \end{tabular*}\vspace{-5pt}
}

\newcommand{\resumeSubSubheading}[2]{
    \begin{tabular*}{0.97\textwidth}{l@{\extracolsep{\fill}}r}
      \textit{\small#1} & \textit{\small #2} \\
    \end{tabular*}\vspace{-5pt}
}

\newcommand{\resumeSubItem}[2]{\resumeItem{#1}{#2}\vspace{-4pt}}

\renewcommand{\labelitemii}{$\circ$}

\newcommand{\resumeSubHeadingListStart}{\begin{itemize}[leftmargin=*]}
\newcommand{\resumeSubHeadingListEnd}{\end{itemize}}
\newcommand{\resumeItemListStart}{\begin{itemize}}
\newcommand{\resumeItemListEnd}{\end{itemize}\vspace{-5pt}}

%-------------------------------------------
%%%%%%  CV STARTS HERE  %%%%%%%%%%%%%%%%%%%%%%%%%%%%


\begin{document}

%----------HEADING-----------------
\begin{tabular*}{\textwidth}{l@{\extracolsep{\fill}}r}
  \textbf{\Large Jerred Shepherd} & Email: \href{mailto:shepherdjerred@gmail.com}{shepherdjerred@gmail.com}\\
  \href{https://shepherdjerred.com/}{https://shepherdjerred.com} & GitHub: \href{https://github.com/shepherdjerred}{https://github.com/shepherdjerred}
\end{tabular*}


%-----------EDUCATION-----------------
\section{Education}
  \resumeSubHeadingListStart
    \resumeSubheading
      {Georgia Tech}{Remote}
      {Master of Science in Computer Science}{August 2022 -- Present}
    \resumeSubheading
      {Harding University}{Searcy, AR}
      {Bachelor of Science in Software Development;  GPA: 3.18}{August 2015 -- May 2019}
  \resumeSubHeadingListEnd


%-----------EXPERIENCE-----------------
\section{Experience}
  \resumeSubHeadingListStart

    \resumeSubheading
      {Posit, PBC (formerly RStudio, PBC)}{Remote}
      {Software Engineer}{September 2021 - Present}
      \resumeItemListStart
        \resumeItem{Automation of end-to-end tests}{Created infrastructure and command-line tooling to automatically run end-to-end tests in on ephemeral machines.}
        \resumeItem{TypeScript support}{Migrated front-end build from Webpack to Vite. Added support for TypeScript, and rewrote state handles in Pinia.}
        \resumeItem{Added support for arm64}{Implemented support for developing Posit Package Manager on the arm64 CPU architecture.}
        \resumeItem{Rewrite of build system}{Analyzed existing build system consisting primarily of Makefiles, Docker, and Jenkins. Rewrote build system to be more understandable, performant, and friendly to developers unfamiliar with the project.}
      \resumeItemListEnd

    \resumeSubheading
      {Amazon Web Services}{Seattle, WA}
      {Software Development Engineer}{July 2019 - August 2021}
      \resumeItemListStart
        \resumeItem{\href{https://aws.amazon.com/about-aws/whats-new/2020/05/aws-systems-manager-now-supports-resource-groups-as-targets-for-state-manager/}{State Manager resource groups feature}}
          {Designed, implemented, tested, and deployed a feature which added support for resource group targets to AWS Systems Manager's desired state configuration service.}
        \resumeItem{Developer productivity improvements}{Developed several productivity tools for use by the State Manager team which led to a significant time savings when developing and deploying code including a notification service for CI/CD events, a service operations report generator, and an infrastructure CLI toolkit.}
        \resumeItem{Intern mentorship}{Mentored an intern during their twelve week internship who was ultimately offered a full-time position. Identified a project and scoped the requirements for the intern. Helped the intern during their onboarding, design, and the development of their project which was deployed to production.}
        \resumeItem{Front end for \href{https://docs.aws.amazon.com/systems-manager/latest/userguide/change-manager.html}{AWS Change Manager}}{Implemented the front-end for AWS Change Manager using React and TypeScript.}
        \resumeItem{Service fleet optimization}
          {Identified and implemented ideal server hardware configuration for team's software stack. Updated service dependencies and language runtimes. These improvements led to a 66\% reduction in server infrastructure cost.}
        \resumeItem{Built service in \href{https://aws.amazon.com/blogs/publicsector/announcing-the-new-aws-secret-region/}{top secret AWS region}}{Modified service code and infrastructure while supporting requirements top secret security constraints.}
        \resumeItem{Infrastructure improvements}{Significantly reduced the time to build a new AWS region for the State Manager service. Identified and implemented process improvements which led to a drastic reduction in operational work.}
        \resumeItemListEnd

    \resumeSubheading
      {Amazon Web Services}{Seattle, WA}
      {Software Development Engineer Intern}{May 2018 - July 2018}
      \resumeItemListStart
        \resumeItem{\href{https://aws.amazon.com/about-aws/whats-new/2019/02/aws-systems-manager-state-manager-enables-document-sharing-across-accounts/}{State Manager document sharing}}
          {Designed, implemented, tested, and deployed a feature which adds cross-account document sharing for AWS Systems Manager's desired state configuration  service.}
      \resumeItemListEnd

  \resumeSubHeadingListEnd


%-----------PROJECTS-----------------
\section{Projects}
  \resumeSubHeadingListStart
    \resumeSubItem{\href{https://github.com/harding-capstone/engine}{Castle Casters}}
      {A cross-platform game and game engine written from scratch in Java 11. Uses OpenGL  for 2D graphics rendering and netty for low-level networking with TCP and UDP sockets. Includes \href{https://github.com/harding-capstone/ai}{an AI} trained with a genetic algorithm and a \href{https://github.com/harding-capstone/logic}{robust implementation} of the \href{https://en.wikipedia.org/wiki/Quoridor}{Quoridor} board game.}
      \resumeSubItem{\href{https://better-skill-capped.shepherdjerred.com/}{Better Skill Capped}}{An improved front-end for the \href{https://www.skill-capped.com/lol}{Skill Capped} website which implements features the original is lacking such as  fuzzy searching, video bookmarking, and offline video viewing. Written using Python, TypeScript, and React. Hosted on AWS with Lambda and S3.}
    \resumeSubItem{\href{https://github.com/shepherdjerred/gpt-2-simple-sagemaker-container}{GPT-2 SageMaker Container}}{Docker image and AWS Lambda Function to train and serve a fine-tined GPT-2 model with AWS SageMaker}
  \resumeSubHeadingListEnd

%-------------------------------------------
\end{document}
